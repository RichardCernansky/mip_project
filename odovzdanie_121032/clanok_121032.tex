% Metódy inžinierskej práce

\documentclass[10pt,twoside,english,a4paper]{article}

\usepackage[english]{babel}
\usepackage[T1]{fontenc}
\usepackage[IL2]{fontenc} % lepšia sadzba písmena Ľ než v T1
\usepackage[utf8]{inputenc}
\usepackage{graphicx}
\usepackage{url} % príkaz \url na formátovanie URL
\usepackage{hyperref} % odkazy v texte budú aktívne (pri niektorých triedach dokumentov spôsobuje posun textu)

\usepackage{cite}
\usepackage{times}
\usepackage{ragged2e}

\pagestyle{headings}

\title{Application of vector algebra and physics in 
designing steering behaviours of autonomous agents for 
realistic display in two dimensions\thanks{Semestrálny projekt v 
predmete Metódy inžinierskej práce, ak. rok 2022/23, 
vedenie: Mirwais Ahmadzai}} 

\author{Richard Čerňanský\\
		Dominik Zaťovič\\[2pt]
	{\small Slovenská technická univerzita v Bratislave}\\
	{\small Fakulta informatiky a informačných technológií}\\
	{\small \texttt{xcernansky@stuba.sk}}\\
	{\small \texttt{xzatovic@stuba.sk}}
	}

\date{\small November 3, 2022 } 

\begin{document}

\maketitle

\begin{abstract}

In this article we will focus on the problematic of animating steering 
behaviors of software-based autonomus agents in 
games. As we want the player to have the best realistic experience
from the movement of the objects possible, we need to design 
their behavior according to the physics that describes it. 
The objective of this aritcle is to explain my ideas on how to calculate 
the steering force in the seek and arrive steering behavior and create 
an algorythm that animates such motion. The algorythm will be written
in p5.js library of JavaScript language. 

\end{abstract}

\section{Introduction}

Development in animation and games forces the creators to make 
the games more and more realistic. One of the most important 
aspects of realistic perception of a game is the movement of 
objects and the animation itself. Jonathan Cooper in the chapter 
\textbf{The 12 principles of animation in video games}\cite{Cooper} 
of his book mentions 12 basic principles of a well-animated game. 
The sixth one named Slow In and Slow Out says:\emph{"Objects that burst into full 
speed immediately can look weightless and unrealistic, so it is here 
again that there is conflict between the gameplay desire to give 
objects ability to move immediately versus the artistic desire to give weight 
to a character."} That is why I came up with some ideas on how to 
design steering motion animations that look as realistic as possible.

In this article, we will call the objects autonomous agents. 
More about them will be explained in part \ref{definition of a.a.}. 
After understanding the concept of autonomous agents, we will state 
a problem in part \ref{problem to solve}. In part \ref{seek and 
arrive} we will have a closer look at the problem and derive some 
formulas necessary for the simulation. Finally, we will simulate 
the steering behavior, stated as a problem in p. \ref{problem to 
solve} by writing the code in p5.js (in p. \ref{code}).

\section{Autonomous agents} \label{autonomous agents}

\subsection{Definition of autonomous agents} \label{definition of a.a.}

The autonomous agents generally refer to an entity that chooses 
how to act in its environment without any influence from a global 
plan or a leader. In games, these agents are not controlled by a 
player, but they are important part of the game because their action
can significantly influence the flow of the game. For example, 
a villain who runs away from police decides on his own that he
wants to escape and starts an action. There are three key 
components of autonomous agents we want to keep in mind 
\cite{Verhagen}. 

\begin{enumerate}
\item An autonomous agent has limited ability to perceive its 
environment. It makes sense that if autonomous agent must decide 
on its action, it should be somehow aware of the environment it 
is located in. The question here is how much limited the ability is. 
If we wanted the object to be all-knowing creature aware of 
everything else around it, we need to give it an access to 
information about everything. On the other side if we wanted it 
to have just very narrow view of the environment, let’s say just 
a few pixels around it.

\item An autonomous agent processes the information from its 
environment and calculates an action. The action is represented 
as a force that influences the object. For example, a police 
officer sees a thief and is attracted to him. The attraction is 
represented by a force pointing towards his location. 

\item An autonomous agent should have no leader. This is not 
something that defines every autonomous agent. Sometimes you need 
to state some global rules that it must follow, but mostly you want 
the object to decide on its own\footnote{however, the designer can 
include some specific atributes if needed}, calculate its own 
actions.

\end{enumerate}

\subsection{Types of behaviors according to the number of objects 
involved} \label{types of behaviors}

Craig Raynolds in his paper from Game Developers Conference 
\textbf{Steering Behaviors For Autonomous Characters} \cite{Raynolds} 
introduces some types of behaviors that could appear talking about 
autonomous agents and how they behave. They divide into two main 
groups:

\begin{itemize}

\item Simple behavior for individuals and pairs\newline
Containing only one or two autonomous agents.

\item Combined behaviors and groups\newline
Containing more than two autonomous agents.

\end{itemize}

In this article we will focus on the first mentioned because 
the combined behaviors are just more complicated kinds but 
the basic ideas are derived from the simple behaviors for 
individuals and groups. 


\section{Problem statement} \label{problem}

Now as we have explained what it takes for the object to behave like 
an autonomous agent we should understand more specific self-operating
concept of a vehicle to be able to grasp on a problem we state later 
in this chapter.

\subsection{Braitenberg vehicle} \label{braitenberg}

Braitenberg vehicle is an entity that is a hypothetical self-operating
machine that can make decisions about how to behave in an environment
based on its sense perception. Valentino Braitenberg explains his 
concept in the book \textbf{Vehicles} \cite{Braitenberg}. Here is 
an example of that type of vehicle from the book:

\bigbreak

\includegraphics[scale=0.33]{braitenberg.jpg}
\quad Figure 1
\bigbreak
Vehicle A steers away from the light source and vehicle B steers 
towards the light source. It is not just about steering away or 
towards the sun. We could say that these vehicles feel emotion about 
the object. Vehicle A feels fear from the sun and vehicle B is 
attracted to it. We want the term vehicle to be clear because in 
the paper when we mention vehicle, we will be referring to the 
concept of Braitenberg vehicle and its characteristics. 

\subsection{Problem to solve} \label{problem to solve}
Our goal is to animate simple steering behavior. The best way to represent
it is by steering a car. It is also a frequently occured situation 
in many games to steer self-controlled car realistically.
Let’s simulate a simple behavior for a pair 
(p. \ref{types of behaviors}). The car that 
is expected to arrive to a specific position (f. e. a parking lot). 
We could say that in the car there is a driver or a device that decides on its own.
It  perceives the environment with its "eyes" and sees the location of the parking 
lot. Processes its distance from the place and calculates and action \ref{definition of a.a.}. 
It starts \textbf{steering} towards the parking lot (attraction towards the 
parking lot is based on the same emotional principle as vehicle B in part \ref{braitenberg}). 
Notice the word steering here which is very important. We do not want 
to simulate an unrealistically moving car that is immediately able to 
turn around and head towards the target in a maximum speed. We want the 
animation to be smooth and realistically looking. What rules do we 
have to follow if we want to simulate a situation like this?

\section{Seek and arrive – a pursuit of a static target} \label{seek and arrive}

In the language of steering behaviors, the problem can be translated as 
a designing a seek and arrival behavior on one object and a target. 
If we want to write an algorythm we need to come up with some 
formulas to calculate the motion.

\subsection{Simple vehicle model} \label{model}
To describe an agent using physics, it is important to have some
mathematical variables that describe it. Craig Raynolds in his 
paper \cite{Raynolds} presents a simple 
vehicle model which needs to be introduced. There are six 
attributes that our vehicle will possess. \\
Simple Vehicle Model: \\
\includegraphics[scale=0.6]{attributes.pdf}

\subsection{Characteristics of seek and arrive } \label{characterictics of seek and arrive}

\subsubsection{Seek} \label{seek}

Seeking steers a character towards a specified position in a space. 
This behavior aligns the vector of a velocity towards the target. 
But how do we calculate the steering force if we do not want the 
vehicle to steer completely right after seeing the target? We can 
diagram our problem from \ref{problem to solve}.

\includegraphics[scale=0.22]{diagram_car.png} 	
\includegraphics[scale=0.22]{diagram_steeringForce.png}\par
\quad Figure 2 
\hspace*{\fill} Figure 3
\bigbreak

The steering force is then calculated with this formula * as shown 
in the Figure 3.

\begin{center}
$desired =position A - position B \quad (vectors)$ \par
$ *steering force= desired – velocity$

\end{center}

As we are designing the animation, the steering force is updated with 
each frame and having lower effect next time each time it is applied. 
The vector of velocity is being aligned with the desired one
more and more with every frame.

\subsubsection{Arrive} \label{arrive}

When we want to simulate arrival, we must think about gradually 
decreasing the speed of the vehicle and eventually stop it once the 
vehicle passes some imaginary boundary. The following diagram describes 
the situation. 

\includegraphics[scale=0.22]{diagram_radius.png}\par
Figure 4
\bigbreak
From the Figure 4 we can assume two things: 

\begin{center}
$d>r 	\Rightarrow speed = maxSpeed$ \par 
$d \leq r \Rightarrow speed = (d/r)  . maxSpeed$ 
\end{center} 

From the second equation we can see that once the object passes through
the circle of the radius r, the speed is decreasing in a fraction of 
the distance $d$ and radius $r$. Once we have the new and \textbf{lower} 
speed calculated, then the magnitude of a vector of desired velocity is set
to the size of that lower max speed. The closer to the place of stopping,
the lower the speed is.

\section{Simulating the behavior in p5.js} \label{simulation} 

The objective of this project was to simulate the behavior. This chapter briefly
explains the tool that was used and describes the main parts of the code.\\
For the link to the animation \textbf{\underline{\href{https://editor.p5js.org/RichardCernansky/sketches/E0zAXshWw}{click here}}}.

\subsection{What is p5.js?} \label{p5 char} 

p5.js is a JavaScript library for creative coding, with a focus on 
making coding accessible and inclusive for artists, designers, or 
educators. It is a collection of pre-written code and provides us 
with tools that simplify the process of creating interactive visuals 
with code in the web browser. p5.js is free and open source. We 
will use it for creating the animation from \ref{problem to solve}. 

\subsection{Main part of the code - class Vehicle and its functions} \label{class Vehicle} 

\subsubsection{constructor() function} \label{constructorf} 

This function creates all the variables for values that were stated in the 
chapter \ref{types of behaviors} (variable for mass is missing because we do not change it in animation
so it will not affect). As an input it gets updated $x$ and $y$ position
of the vehicle. 

\subsubsection{seek() function} \label{seekf} 

Seek function gets as an input vector Target representing $x$ and $y$ position
of the target. First, it calculates the steering force by subtracting velocity vector
from desired velocity vector (\ref{seek}). An $if$ statement included in a function checks if 
the distance is lower than 100 pixels. If yes, then it sets the magnitude of the 
desired velocity to some portion of max speed accordingly to the distance. 
Otherwise, it sets the magnitude to the size of max speed.

\subsubsection{update() function} \label{updatef} 

Update function applies the steering force combined with velocity to the vehicle
every frame. The result is our realistically looking steering animation. 

\section{Research method and literature review} \label{research method} 

The source article \cite{Raynolds} is what introduced me to the topic. 
From the beginning of my research there were a lot of questions. 
The idea of autonomus agent was mentioned in Raynold's article
but to completely understand it, I had to search deeper. A part of the book Vehicles 
\cite{Braitenberg} and the article from Verhangen \cite{Verhagen} gave me the complete
understanding. 

Raynold's article is what also helped me the most with coming up with formulas for
calculation of the steering force. The explanation of his ideas is clear and 
understandable for anyone with a knowledge of coordinate geometry 
and physics of forces. 

\section{Results} \label{results} 

After stating a problem in \ref{problem to solve}, the result of this 
project was made clear. I created a realistic animation of steering bahavior
using knowledge of forces, vectors and operating with them. 
This can be used in game graphics for creating self-operating worlds
of objects that move according to an algorythm they obey. 

\section{Conclusion} \label{conclusion}

To sum up, the goal of this project was to animate seek and arrival 
steering behavior in two dimensions. The challange was to 
make it look as realitic as possible. Coming 
up with the formulas to calculate the steering force using 
physics was necessary. We did not want the vehicle to be able to 
steer immediately and rush in the full speed to the target. Instead, 
we wanted to make the animation smooth, which was done. 

The second challenge was to make the vehicle eventually stop after 
reaching the target. The idea of gradually slowing down after 
crossing a boundary works perfectly. The last part was to put the ideas
for steering into an algorythm that animated such a behavior. 

Although I succeeded in creating a certain type of steering behavior,
there are still many to be invented. For example, complicated group behaviors
follow-up this topic. Also considering the three dimensional space
and inventing pure reality-based models is still an open topic to be explored
and used both in games and reality for motion behaviors of 
self-driving cars.

\section*{Acknowledgement}
I would like to thank my friend Lukáš Častven for introducing 
the problem to me and arousing my interest in the topic\ldots

\bibliography{literatura}
\bibliographystyle{plain} 

\end{document}

